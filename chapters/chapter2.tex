\documentclass[../Syllabus.tex]{subfiles}

\begin{document}

\section{Chapitre 2 : Class Diagram}

\subsection{Motivation}

Un Class Diagram exprime les charactéristiques statiques des différents éléments d'un Système d'Information. Il tire son origine du paradigme de programmation Orienté Objet. Les spécifications du programme final dérivent du modèle ainsi construit.

Lors de la conception du modèle, le système sera décomposé en une multitude de petits composants. Cette approche permettre un gain considérable au niveau de la complexité, mais aussi au niveau des utilisations futures. En effet, les différents concepts générés pourront être réutilisés pendant tout le développement et même durant d'autres projets. D'autre part, le modèle est plus facilement modifiable qu'un code source. On peut progressivement modifier le modelling pour le faire correspondre progressivement aux attentes.

\subsection{Caractéristiques}

Un objet représente un élément possédant

\begin{itemize}
  \item Une \textbf{identité} : l'élément existe dans le système modélisé.
  \item Une \textbf{durée de vie} : l'élément existe pour une période donnée.
  \item Un \textbf{état} : l'élément possède des caractéristiques à un moment donné.
  \item Une \textbf{pertinence} et une \textbf{signification} : tout deux établis dans le domaine et décris dans le glossaire.
\end{itemize}

\subsection{Modélisation}

Le Système d'Information peut être composé de millions de différents éléments qui ne peuvent pas être représenté de manière individuelle dans le modèle. Les Class permettent de décomposer ce monde en un ensemble d'objets partageant des caractéristiques communes et pertinentes.

% TODO PAGE160 On doit vraiment parler de ça ?

\end{document}
